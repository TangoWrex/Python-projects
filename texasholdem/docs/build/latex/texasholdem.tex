%% Generated by Sphinx.
\def\sphinxdocclass{report}
\documentclass[letterpaper,10pt,english]{sphinxmanual}
\ifdefined\pdfpxdimen
   \let\sphinxpxdimen\pdfpxdimen\else\newdimen\sphinxpxdimen
\fi \sphinxpxdimen=.75bp\relax
\ifdefined\pdfimageresolution
    \pdfimageresolution= \numexpr \dimexpr1in\relax/\sphinxpxdimen\relax
\fi
%% let collapsable pdf bookmarks panel have high depth per default
\PassOptionsToPackage{bookmarksdepth=5}{hyperref}

\PassOptionsToPackage{warn}{textcomp}
\usepackage[utf8]{inputenc}
\ifdefined\DeclareUnicodeCharacter
% support both utf8 and utf8x syntaxes
  \ifdefined\DeclareUnicodeCharacterAsOptional
    \def\sphinxDUC#1{\DeclareUnicodeCharacter{"#1}}
  \else
    \let\sphinxDUC\DeclareUnicodeCharacter
  \fi
  \sphinxDUC{00A0}{\nobreakspace}
  \sphinxDUC{2500}{\sphinxunichar{2500}}
  \sphinxDUC{2502}{\sphinxunichar{2502}}
  \sphinxDUC{2514}{\sphinxunichar{2514}}
  \sphinxDUC{251C}{\sphinxunichar{251C}}
  \sphinxDUC{2572}{\textbackslash}
\fi
\usepackage{cmap}
\usepackage[T1]{fontenc}
\usepackage{amsmath,amssymb,amstext}
\usepackage{babel}



\usepackage{tgtermes}
\usepackage{tgheros}
\renewcommand{\ttdefault}{txtt}



\usepackage[Bjarne]{fncychap}
\usepackage{sphinx}

\fvset{fontsize=auto}
\usepackage{geometry}


% Include hyperref last.
\usepackage{hyperref}
% Fix anchor placement for figures with captions.
\usepackage{hypcap}% it must be loaded after hyperref.
% Set up styles of URL: it should be placed after hyperref.
\urlstyle{same}

\addto\captionsenglish{\renewcommand{\contentsname}{Contents:}}

\usepackage{sphinxmessages}
\setcounter{tocdepth}{1}



\title{texasholdem}
\date{Sep 08, 2021}
\release{}
\author{mark truett}
\newcommand{\sphinxlogo}{\vbox{}}
\renewcommand{\releasename}{}
\makeindex
\begin{document}

\pagestyle{empty}
\sphinxmaketitle
\pagestyle{plain}
\sphinxtableofcontents
\pagestyle{normal}
\phantomsection\label{\detokenize{index::doc}}



\chapter{Operations}
\label{\detokenize{readme:operations}}\label{\detokenize{readme::doc}}

\section{\sphinxstylestrong{Purpose:}}
\label{\detokenize{readme:purpose}}
\sphinxAtStartPar
Texas holdem is a 1 vs 1\sphinxhyphen{}8 computer player game.
the game randomly generates your cards. you’re able to to bet
\$1 \sphinxhyphen{} \$1000. at the end of the game all cards are shown and a winner
is returned


\section{\sphinxstylestrong{challenges and successes:}}
\label{\detokenize{readme:challenges-and-successes}}
\sphinxAtStartPar
I was successful with creating my deck and creating my players in a dictionary
I was able to accomplish that fairly quickly and felt confident in using functions to pass the data
throughout my program.

\sphinxAtStartPar
it was very challenging to use the lambdas and combination to check the different combinations
against the functions in RankTheHand.py.

\sphinxAtStartPar
Unfortunately, I could not figure out how toget my program to pick a winner.


\section{\sphinxstylestrong{Link}}
\label{\detokenize{readme:link}}
\sphinxAtStartPar
{[}This is a link to gitlab{]}(\sphinxurl{https://git.cybbh.space/170D/wobc/student-folders/21\_002/truett/texasholdem/-/blob/master/})


\section{\sphinxstylestrong{functions used in operations}}
\label{\detokenize{readme:functions-used-in-operations}}
\sphinxAtStartPar
functions used in texasholdem

\sphinxAtStartPar
gen\_player\_hand ()

\sphinxAtStartPar
generate\_the\_flop()

\sphinxAtStartPar
generate\_a\_card()

\sphinxAtStartPar
generate\_deck()

\sphinxAtStartPar
create\_players()

\sphinxAtStartPar
handle\_a\_bet()

\sphinxAtStartPar
the\_hand\_is\_a()

\sphinxAtStartPar
return\_all\_players\_\_best\_hand()

\sphinxAtStartPar
main()


\chapter{PlayHoldEm}
\label{\detokenize{modules:playholdem}}\label{\detokenize{modules::doc}}

\section{PlayHoldEm module}
\label{\detokenize{PlayHoldEm:module-PlayHoldEm}}\label{\detokenize{PlayHoldEm:playholdem-module}}\label{\detokenize{PlayHoldEm::doc}}\index{module@\spxentry{module}!PlayHoldEm@\spxentry{PlayHoldEm}}\index{PlayHoldEm@\spxentry{PlayHoldEm}!module@\spxentry{module}}
\sphinxAtStartPar
PlayHoldEm Game.

\sphinxAtStartPar
Texas holdem is a 1 vs 1\sphinxhyphen{}8 computer player game.
the game randomly generates your cards. you’re able to to bet
1 \sphinxhyphen{} 1000. at the end of the game all cards are shown and a winner
is returned
\index{create\_players() (in module PlayHoldEm)@\spxentry{create\_players()}\spxextra{in module PlayHoldEm}}

\begin{fulllineitems}
\phantomsection\label{\detokenize{PlayHoldEm:PlayHoldEm.create_players}}\pysiglinewithargsret{\sphinxcode{\sphinxupquote{PlayHoldEm.}}\sphinxbfcode{\sphinxupquote{create\_players}}}{\emph{\DUrole{n}{the\_deck}}, \emph{\DUrole{n}{num\_players}}}{}
\sphinxAtStartPar
create\_players.

\sphinxAtStartPar
a dictionary of players are created and given 2 cards,
\$1000. the amount of players is input by the user.

\end{fulllineitems}

\index{gen\_player\_hand() (in module PlayHoldEm)@\spxentry{gen\_player\_hand()}\spxextra{in module PlayHoldEm}}

\begin{fulllineitems}
\phantomsection\label{\detokenize{PlayHoldEm:PlayHoldEm.gen_player_hand}}\pysiglinewithargsret{\sphinxcode{\sphinxupquote{PlayHoldEm.}}\sphinxbfcode{\sphinxupquote{gen\_player\_hand}}}{\emph{\DUrole{n}{the\_deck}}, \emph{\DUrole{n}{number\_of\_cards}}}{}
\sphinxAtStartPar
Gen\_player\_hand.

\sphinxAtStartPar
Generates Players Hand
the number of cards are how many cards are to be returned
those cards are removed to only be used once.

\end{fulllineitems}

\index{generate\_a\_card() (in module PlayHoldEm)@\spxentry{generate\_a\_card()}\spxextra{in module PlayHoldEm}}

\begin{fulllineitems}
\phantomsection\label{\detokenize{PlayHoldEm:PlayHoldEm.generate_a_card}}\pysiglinewithargsret{\sphinxcode{\sphinxupquote{PlayHoldEm.}}\sphinxbfcode{\sphinxupquote{generate\_a\_card}}}{\emph{\DUrole{n}{the\_deck}}}{}
\sphinxAtStartPar
generate\_a\_card.

\sphinxAtStartPar
Generates 1 card. the 2nd and third round.

\end{fulllineitems}

\index{generate\_deck() (in module PlayHoldEm)@\spxentry{generate\_deck()}\spxextra{in module PlayHoldEm}}

\begin{fulllineitems}
\phantomsection\label{\detokenize{PlayHoldEm:PlayHoldEm.generate_deck}}\pysiglinewithargsret{\sphinxcode{\sphinxupquote{PlayHoldEm.}}\sphinxbfcode{\sphinxupquote{generate\_deck}}}{}{}
\sphinxAtStartPar
generate\_deck.

\sphinxAtStartPar
This function creates the deck

\end{fulllineitems}

\index{generate\_the\_flop() (in module PlayHoldEm)@\spxentry{generate\_the\_flop()}\spxextra{in module PlayHoldEm}}

\begin{fulllineitems}
\phantomsection\label{\detokenize{PlayHoldEm:PlayHoldEm.generate_the_flop}}\pysiglinewithargsret{\sphinxcode{\sphinxupquote{PlayHoldEm.}}\sphinxbfcode{\sphinxupquote{generate\_the\_flop}}}{\emph{\DUrole{n}{the\_deck}}}{}
\sphinxAtStartPar
generate\_the\_flop.

\sphinxAtStartPar
Generates 3 cards

\sphinxAtStartPar
These are the first three community cards

\end{fulllineitems}

\index{handle\_a\_bet() (in module PlayHoldEm)@\spxentry{handle\_a\_bet()}\spxextra{in module PlayHoldEm}}

\begin{fulllineitems}
\phantomsection\label{\detokenize{PlayHoldEm:PlayHoldEm.handle_a_bet}}\pysiglinewithargsret{\sphinxcode{\sphinxupquote{PlayHoldEm.}}\sphinxbfcode{\sphinxupquote{handle\_a\_bet}}}{\emph{\DUrole{n}{a\_player}}, \emph{\DUrole{n}{amount\_to\_bet}}, \emph{\DUrole{n}{the\_pot}}}{}
\sphinxAtStartPar
handle\_a\_bet.

\sphinxAtStartPar
Player 1s bet is handlded here and counted in the pot

\end{fulllineitems}

\index{handle\_other\_bets() (in module PlayHoldEm)@\spxentry{handle\_other\_bets()}\spxextra{in module PlayHoldEm}}

\begin{fulllineitems}
\phantomsection\label{\detokenize{PlayHoldEm:PlayHoldEm.handle_other_bets}}\pysiglinewithargsret{\sphinxcode{\sphinxupquote{PlayHoldEm.}}\sphinxbfcode{\sphinxupquote{handle\_other\_bets}}}{\emph{\DUrole{n}{a\_player}}, \emph{\DUrole{n}{amount\_bet}}, \emph{\DUrole{n}{the\_pot}}}{}
\sphinxAtStartPar
handle\_other\_bets.

\sphinxAtStartPar
This function brings amount\_bet from player 1 and applied that to the
rest of the players

\end{fulllineitems}

\index{main() (in module PlayHoldEm)@\spxentry{main()}\spxextra{in module PlayHoldEm}}

\begin{fulllineitems}
\phantomsection\label{\detokenize{PlayHoldEm:PlayHoldEm.main}}\pysiglinewithargsret{\sphinxcode{\sphinxupquote{PlayHoldEm.}}\sphinxbfcode{\sphinxupquote{main}}}{}{}
\end{fulllineitems}

\index{return\_all\_players\_best\_hand() (in module PlayHoldEm)@\spxentry{return\_all\_players\_best\_hand()}\spxextra{in module PlayHoldEm}}

\begin{fulllineitems}
\phantomsection\label{\detokenize{PlayHoldEm:PlayHoldEm.return_all_players_best_hand}}\pysiglinewithargsret{\sphinxcode{\sphinxupquote{PlayHoldEm.}}\sphinxbfcode{\sphinxupquote{return\_all\_players\_best\_hand}}}{\emph{\DUrole{n}{player\_list}}}{}
\sphinxAtStartPar
return\_all\_players\_best\_hand.

\sphinxAtStartPar
we iterate through each players best hand.
each players hand is ran through the players best hand
and the highest value is returned to pick\_a\_winner

\end{fulllineitems}

\index{the\_hand\_is\_a() (in module PlayHoldEm)@\spxentry{the\_hand\_is\_a()}\spxextra{in module PlayHoldEm}}

\begin{fulllineitems}
\phantomsection\label{\detokenize{PlayHoldEm:PlayHoldEm.the_hand_is_a}}\pysiglinewithargsret{\sphinxcode{\sphinxupquote{PlayHoldEm.}}\sphinxbfcode{\sphinxupquote{the\_hand\_is\_a}}}{\emph{\DUrole{n}{player\_hand}}}{}
\sphinxAtStartPar
the\_hand\_is\_a.

\sphinxAtStartPar
the hand is uses the rankthehand.py file to rank each hand

\end{fulllineitems}

\index{the\_players\_best\_hand() (in module PlayHoldEm)@\spxentry{the\_players\_best\_hand()}\spxextra{in module PlayHoldEm}}

\begin{fulllineitems}
\phantomsection\label{\detokenize{PlayHoldEm:PlayHoldEm.the_players_best_hand}}\pysiglinewithargsret{\sphinxcode{\sphinxupquote{PlayHoldEm.}}\sphinxbfcode{\sphinxupquote{the\_players\_best\_hand}}}{\emph{\DUrole{n}{player\_hand}}}{}
\sphinxAtStartPar
the\_players\_best\_hand.

\sphinxAtStartPar
We take all possible combinations of the players hand and run it
against rankthehand.py.
we then sort the tuple

\end{fulllineitems}



\section{RankTheHand module}
\label{\detokenize{RankTheHand:module-RankTheHand}}\label{\detokenize{RankTheHand:rankthehand-module}}\label{\detokenize{RankTheHand::doc}}\index{module@\spxentry{module}!RankTheHand@\spxentry{RankTheHand}}\index{RankTheHand@\spxentry{RankTheHand}!module@\spxentry{module}}\index{cards\_all\_same\_suit() (in module RankTheHand)@\spxentry{cards\_all\_same\_suit()}\spxextra{in module RankTheHand}}

\begin{fulllineitems}
\phantomsection\label{\detokenize{RankTheHand:RankTheHand.cards_all_same_suit}}\pysiglinewithargsret{\sphinxcode{\sphinxupquote{RankTheHand.}}\sphinxbfcode{\sphinxupquote{cards\_all\_same\_suit}}}{\emph{\DUrole{n}{a\_hand}}}{}
\end{fulllineitems}

\index{is\_flush() (in module RankTheHand)@\spxentry{is\_flush()}\spxextra{in module RankTheHand}}

\begin{fulllineitems}
\phantomsection\label{\detokenize{RankTheHand:RankTheHand.is_flush}}\pysiglinewithargsret{\sphinxcode{\sphinxupquote{RankTheHand.}}\sphinxbfcode{\sphinxupquote{is\_flush}}}{\emph{\DUrole{n}{a\_hand}}}{}
\end{fulllineitems}

\index{is\_four\_kind() (in module RankTheHand)@\spxentry{is\_four\_kind()}\spxextra{in module RankTheHand}}

\begin{fulllineitems}
\phantomsection\label{\detokenize{RankTheHand:RankTheHand.is_four_kind}}\pysiglinewithargsret{\sphinxcode{\sphinxupquote{RankTheHand.}}\sphinxbfcode{\sphinxupquote{is\_four\_kind}}}{\emph{\DUrole{n}{a\_hand}}}{}
\end{fulllineitems}

\index{is\_full\_house() (in module RankTheHand)@\spxentry{is\_full\_house()}\spxextra{in module RankTheHand}}

\begin{fulllineitems}
\phantomsection\label{\detokenize{RankTheHand:RankTheHand.is_full_house}}\pysiglinewithargsret{\sphinxcode{\sphinxupquote{RankTheHand.}}\sphinxbfcode{\sphinxupquote{is\_full\_house}}}{\emph{\DUrole{n}{a\_hand}}}{}
\end{fulllineitems}

\index{is\_pair() (in module RankTheHand)@\spxentry{is\_pair()}\spxextra{in module RankTheHand}}

\begin{fulllineitems}
\phantomsection\label{\detokenize{RankTheHand:RankTheHand.is_pair}}\pysiglinewithargsret{\sphinxcode{\sphinxupquote{RankTheHand.}}\sphinxbfcode{\sphinxupquote{is\_pair}}}{\emph{\DUrole{n}{a\_hand}}}{}
\end{fulllineitems}

\index{is\_straight() (in module RankTheHand)@\spxentry{is\_straight()}\spxextra{in module RankTheHand}}

\begin{fulllineitems}
\phantomsection\label{\detokenize{RankTheHand:RankTheHand.is_straight}}\pysiglinewithargsret{\sphinxcode{\sphinxupquote{RankTheHand.}}\sphinxbfcode{\sphinxupquote{is\_straight}}}{\emph{\DUrole{n}{a\_hand}}}{}
\end{fulllineitems}

\index{is\_straight\_flush() (in module RankTheHand)@\spxentry{is\_straight\_flush()}\spxextra{in module RankTheHand}}

\begin{fulllineitems}
\phantomsection\label{\detokenize{RankTheHand:RankTheHand.is_straight_flush}}\pysiglinewithargsret{\sphinxcode{\sphinxupquote{RankTheHand.}}\sphinxbfcode{\sphinxupquote{is\_straight\_flush}}}{\emph{\DUrole{n}{a\_hand}}}{}
\end{fulllineitems}

\index{is\_three\_kind() (in module RankTheHand)@\spxentry{is\_three\_kind()}\spxextra{in module RankTheHand}}

\begin{fulllineitems}
\phantomsection\label{\detokenize{RankTheHand:RankTheHand.is_three_kind}}\pysiglinewithargsret{\sphinxcode{\sphinxupquote{RankTheHand.}}\sphinxbfcode{\sphinxupquote{is\_three\_kind}}}{\emph{\DUrole{n}{a\_hand}}}{}
\end{fulllineitems}

\index{is\_two\_pairs() (in module RankTheHand)@\spxentry{is\_two\_pairs()}\spxextra{in module RankTheHand}}

\begin{fulllineitems}
\phantomsection\label{\detokenize{RankTheHand:RankTheHand.is_two_pairs}}\pysiglinewithargsret{\sphinxcode{\sphinxupquote{RankTheHand.}}\sphinxbfcode{\sphinxupquote{is\_two\_pairs}}}{\emph{\DUrole{n}{a\_hand}}}{}
\end{fulllineitems}

\index{return\_cards\_rank\_as\_straight() (in module RankTheHand)@\spxentry{return\_cards\_rank\_as\_straight()}\spxextra{in module RankTheHand}}

\begin{fulllineitems}
\phantomsection\label{\detokenize{RankTheHand:RankTheHand.return_cards_rank_as_straight}}\pysiglinewithargsret{\sphinxcode{\sphinxupquote{RankTheHand.}}\sphinxbfcode{\sphinxupquote{return\_cards\_rank\_as\_straight}}}{\emph{\DUrole{n}{a\_hand}}}{}
\end{fulllineitems}

\index{return\_most\_frequent\_and\_num\_counter\_elements() (in module RankTheHand)@\spxentry{return\_most\_frequent\_and\_num\_counter\_elements()}\spxextra{in module RankTheHand}}

\begin{fulllineitems}
\phantomsection\label{\detokenize{RankTheHand:RankTheHand.return_most_frequent_and_num_counter_elements}}\pysiglinewithargsret{\sphinxcode{\sphinxupquote{RankTheHand.}}\sphinxbfcode{\sphinxupquote{return\_most\_frequent\_and\_num\_counter\_elements}}}{\emph{\DUrole{n}{a\_hand}}}{}
\end{fulllineitems}



\section{validator module}
\label{\detokenize{validator:module-validator}}\label{\detokenize{validator:validator-module}}\label{\detokenize{validator::doc}}\index{module@\spxentry{module}!validator@\spxentry{validator}}\index{validator@\spxentry{validator}!module@\spxentry{module}}
\sphinxAtStartPar
Module that validates user input
Has various functions that check integer, float,
is number in range, is entry contained in

\sphinxAtStartPar
Functions continue to prompt until input
is valid
\index{enter\_a\_float() (in module validator)@\spxentry{enter\_a\_float()}\spxextra{in module validator}}

\begin{fulllineitems}
\phantomsection\label{\detokenize{validator:validator.enter_a_float}}\pysiglinewithargsret{\sphinxcode{\sphinxupquote{validator.}}\sphinxbfcode{\sphinxupquote{enter\_a\_float}}}{\emph{\DUrole{n}{prompt}}}{}
\end{fulllineitems}

\index{enter\_an\_integer() (in module validator)@\spxentry{enter\_an\_integer()}\spxextra{in module validator}}

\begin{fulllineitems}
\phantomsection\label{\detokenize{validator:validator.enter_an_integer}}\pysiglinewithargsret{\sphinxcode{\sphinxupquote{validator.}}\sphinxbfcode{\sphinxupquote{enter\_an\_integer}}}{\emph{\DUrole{n}{prompt}}}{}
\end{fulllineitems}

\index{enter\_correct\_type() (in module validator)@\spxentry{enter\_correct\_type()}\spxextra{in module validator}}

\begin{fulllineitems}
\phantomsection\label{\detokenize{validator:validator.enter_correct_type}}\pysiglinewithargsret{\sphinxcode{\sphinxupquote{validator.}}\sphinxbfcode{\sphinxupquote{enter\_correct\_type}}}{\emph{\DUrole{n}{prompt}}, \emph{\DUrole{n}{type\_function}}}{}
\end{fulllineitems}

\index{enter\_float\_in\_range() (in module validator)@\spxentry{enter\_float\_in\_range()}\spxextra{in module validator}}

\begin{fulllineitems}
\phantomsection\label{\detokenize{validator:validator.enter_float_in_range}}\pysiglinewithargsret{\sphinxcode{\sphinxupquote{validator.}}\sphinxbfcode{\sphinxupquote{enter\_float\_in\_range}}}{\emph{\DUrole{n}{prompt}}, \emph{\DUrole{n}{low}}, \emph{\DUrole{n}{high}}}{}
\end{fulllineitems}

\index{enter\_integer\_in\_range() (in module validator)@\spxentry{enter\_integer\_in\_range()}\spxextra{in module validator}}

\begin{fulllineitems}
\phantomsection\label{\detokenize{validator:validator.enter_integer_in_range}}\pysiglinewithargsret{\sphinxcode{\sphinxupquote{validator.}}\sphinxbfcode{\sphinxupquote{enter\_integer\_in\_range}}}{\emph{\DUrole{n}{prompt}}, \emph{\DUrole{n}{low}}, \emph{\DUrole{n}{high}}}{}
\end{fulllineitems}

\index{enter\_number\_in\_range() (in module validator)@\spxentry{enter\_number\_in\_range()}\spxextra{in module validator}}

\begin{fulllineitems}
\phantomsection\label{\detokenize{validator:validator.enter_number_in_range}}\pysiglinewithargsret{\sphinxcode{\sphinxupquote{validator.}}\sphinxbfcode{\sphinxupquote{enter\_number\_in\_range}}}{\emph{\DUrole{n}{prompt}}, \emph{\DUrole{n}{low}}, \emph{\DUrole{n}{high}}, \emph{\DUrole{n}{entry\_function}}}{}
\end{fulllineitems}

\index{enter\_valid\_character() (in module validator)@\spxentry{enter\_valid\_character()}\spxextra{in module validator}}

\begin{fulllineitems}
\phantomsection\label{\detokenize{validator:validator.enter_valid_character}}\pysiglinewithargsret{\sphinxcode{\sphinxupquote{validator.}}\sphinxbfcode{\sphinxupquote{enter\_valid\_character}}}{\emph{\DUrole{n}{prompt}}, \emph{\DUrole{n}{set\_of\_valid\_chars}\DUrole{p}{:} \DUrole{n}{str}}, \emph{\DUrole{n}{ignore\_case}\DUrole{o}{=}\DUrole{default_value}{True}}}{}
\end{fulllineitems}

\index{main() (in module validator)@\spxentry{main()}\spxextra{in module validator}}

\begin{fulllineitems}
\phantomsection\label{\detokenize{validator:validator.main}}\pysiglinewithargsret{\sphinxcode{\sphinxupquote{validator.}}\sphinxbfcode{\sphinxupquote{main}}}{}{}
\end{fulllineitems}



\chapter{Indices and tables}
\label{\detokenize{index:indices-and-tables}}\begin{itemize}
\item {} 
\sphinxAtStartPar
\DUrole{xref,std,std-ref}{genindex}

\item {} 
\sphinxAtStartPar
\DUrole{xref,std,std-ref}{modindex}

\item {} 
\sphinxAtStartPar
\DUrole{xref,std,std-ref}{search}

\end{itemize}


\renewcommand{\indexname}{Python Module Index}
\begin{sphinxtheindex}
\let\bigletter\sphinxstyleindexlettergroup
\bigletter{p}
\item\relax\sphinxstyleindexentry{PlayHoldEm}\sphinxstyleindexpageref{PlayHoldEm:\detokenize{module-PlayHoldEm}}
\indexspace
\bigletter{r}
\item\relax\sphinxstyleindexentry{RankTheHand}\sphinxstyleindexpageref{RankTheHand:\detokenize{module-RankTheHand}}
\indexspace
\bigletter{v}
\item\relax\sphinxstyleindexentry{validator}\sphinxstyleindexpageref{validator:\detokenize{module-validator}}
\end{sphinxtheindex}

\renewcommand{\indexname}{Index}
\printindex
\end{document}